\begin{frame}{}
	\centering \textbf{\huge A short review..}
\end{frame}

\begin{frame}{Maximum Matching vs Maximal Matching}
	\parbox[c][.7\textheight][c]{\textwidth}{%
\only<1>
{
	\begin{center}
    \includegraphics[width=0.15\linewidth]{img/random/maximum-maximal-0.eps}
	\end{center}
}
\only<2>
{
	\begin{center}
    \includegraphics[width=0.15\linewidth]{img/random/maximum-maximal-1.eps}
	\end{center}
}
\only<3->
{
	\begin{center}
    \includegraphics[width=0.15\linewidth]{img/random/maximum-maximal-2.eps}
	\end{center}
}
\uncover<4>
{
	\begin{center}
    \includegraphics[width=0.15\linewidth]{img/random/maximum-maximal-3.eps}
	\end{center}
}
}
\end{frame}

\begin{frame}[t]{Determinant and Permanent}
	A short trip in mathematics,\\
	\textbf{Permutation} - Bijection over $n$ elements.\\
	$\mathcal{S}_n$ - Set of all permutatoins of $n$ elements.\\~\\
	For an $n\times n$-matrix $A$, we define

\begin{columns}[T] % align columns\end{column}%
\begin{column}{.48\textwidth}
	{\centering \textbf{Determinant of a matrix}\\}
$$det(A) =  \sum\limits_{\pi \in \mathcal{S}} sign(\pi) \prod\limits_{i\in[n]} A_{i, \pi(i)}$$
	laplace expansion (on board).\\
	Efficient Gaussian Eliminatin.\\
	$O(n^\omega), \omega = 2.373$.\\
	Matrix-Multiplication Exponent.\\

\end{column}
\hfill%
\begin{column}{.48\textwidth}
	{\centering \textbf{Permanent of a matrix}\\}
$$perm(A) =  \sum\limits_{\pi \in \mathcal{S}}  \prod\limits_{i\in[n]} A_{i, \pi(i)}$$
	\textbf{NP-Hard} problem.\\
	Presumably no polynomial time algorithm to compute.\\

\end{column}
\end{columns}
\end{frame}

\begin{frame}[t]{Randomized Perfect Matching}
	How to check if a bipartite graph admits a perfect matching?\\

\begin{tabular}{cl}  
\begin{tabular}{c}
\only<1-2>
{
    \includegraphics[width=0.15\linewidth]{img/random/randombip-orig.eps}
}
\only<3->
{
    \includegraphics[width=0.15\linewidth]{img/random/randombip-matched.eps}
}
\end{tabular}
& \begin{tabular}{l}
\parbox{0.7\linewidth}{%  change the parbox width as appropiate
	\only<2-3>
	{
	\[
	A^G=
	\begin{bmatrix}
	1 & 1 & 0 \\
	0 & 1 & 1 \\
	1 & 0 & 0 
	\end{bmatrix}
	\]
	}
	\only<4->
	{
	\[
	A^G=
	\begin{bmatrix}
	1 & \textcolor{red}{1} & 0 \\
	0 & 1 & \textcolor{red}{1} \\
	\textcolor{red}{1} & 0 & 0 
	\end{bmatrix}
	\]
	}

\uncover<5->
{
	\begin{center}
	\begin{tabular}{|l|*{3}{c}|r}
		\hline
		$u_i$ & 1 &  2 & 3  \\
		\hline
		$\pi(u_i)$ & 2 &  3 & 1  \\
		\hline
	\end{tabular}
	\end{center}
}
}
\\
\end{tabular}
\end{tabular}
\only<6->
{
	The graph admits a perfect matching $\iff$\\
}
\only<7->
{
	There is a permutation $\pi$, s.t. $\prod\limits_{i \in [n]}A^G_{i, \pi(i)} = 1$  $\iff$\\
}
\only<8>
{
	For $\mathcal{S}$ the set of all permutations on $n$ elements
\[
	\sum\limits_{\pi \in \mathcal{S} } \prod\limits_{i \in [n]} A^G_{i, \pi(i)} > 0
\]
}
\only<9->
{
For $\mathcal{S}$ the set of all permutations on $n$ elements
\[
	\underbrace{\sum\limits_{\pi \in \mathcal{S} } \prod\limits_{i \in [n]} A^G_{i, \pi(i)}}_{perm(A^G)} > 0
\]
}
\end{frame}

\begin{frame}[t]{Permanent and Determinant}
	\begin{align*}
		perm(A) = & \sum\limits_{\pi \in \mathcal{S}} \prod\limits_{i\in[n]} A_{i, \pi(i)} & \uncover<2->{\dots\text{is a known hard problem}}\\
		\uncover<3->{det(A) = & \sum\limits_{\pi \in \mathcal{S}} sign(\pi) \prod\limits_{i\in[n]} A_{i, \pi(i)} & }
	\end{align*}

	\uncover<4->{We don't have to bound ourselves with ones in the matrix..\\}
	\uncover<5->{
	\begin{tabular}{cl}  
	\begin{tabular}{c}
    	\includegraphics[width=0.15\linewidth]{img/random/randombip-orig.eps}
	\end{tabular}
	\begin{tabular}{l}
	\parbox{0.7\linewidth}{%  change the parbox width as appropiate
	\[
	A'^G=
	\begin{bmatrix}
		x_{11} & x_{12} & 0 \\
		0 & x_{22} & x_{23} \\
		x_{31} & 0 & 0 
	\end{bmatrix}
	\]
	}
	\end{tabular}
	\end{tabular}
	}
	\uncover<6->{$det(A'^G)$ is a polynomial of degree $n$.\\}
	\uncover<7->{The graph admits a perfect matching $\iff$ the polynomial $det(A'^G)$ is not identical zero.}
\end{frame}

\begin{frame}[t]{Schwartz-Zippel lemma} 
	\begin{block}{Shwartz-Zippel lemma}
		Let $\rho$ be a non-zero polynomial of $n$ variables and degree $d$ over a field $\mathbb{F}$.\\
		Let $\mathcal{S} \subseteq \mathbb{F}^n$, then for $x_0 \underset{u.a.r}{\in} \mathcal{S}$
		$$Pr[\rho(x_0) = 0] \leq \frac{d}{|\mathcal{S}|}$$
	\end{block}
	\only<2->
	{
	Choosing $|\mathcal{S}| = 2d$, we get a method that tells with probability at least $1/2$, if the polynomial is identical zero.\\
	}
	\only<3->
	{
	Since one non-zero answer is enough to know the polynomial is a non-zero, we can repeat the operation a couple of times magnifying the probability.\\
	}
\end{frame}
